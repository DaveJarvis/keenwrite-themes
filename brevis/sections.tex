% %%%%%%%%%%%%%%%%%%%%%%%%%%%%%%%%%%%%%%%%%%%%%%%%%%%%%%%%%%%%%%%%%%%%%%%%%%% 
%
% Configure presentation for document section content.
%
%%%%%%%%%%%%%%%%%%%%%%%%%%%%%%%%%%%%%%%%%%%%%%%%%%%%%%%%%%%%%%%%%%%%%%%%%%%%

% Rounded rectangle surrounds the summary text.
\startuseMPgraphic{TextBackgroundGraphic}
  begingroup ;
    for i = 1 upto nofmultipars :
      draw (
        llcorner multipars[ i ] --
        lrcorner multipars[ i ] --
        urcorner multipars[ i ] --
        ulcorner multipars[ i ] --
        cycle
      )
      enlarged( EmWidth, EmWidth )
      cornered( 2 * EmWidth )
      withcolor "darkgray" ;
    endfor ;
  endgroup ;
\stopuseMPgraphic

\definetextbackground[TextBackground][
  location=paragraph,
  mp=TextBackgroundGraphic,
  before={\blank[2*big]},
]

% Enumerate the overview (storylines).
\definenumber[TextCounterOverview][
  prefix=no,
]

\setnumber[TextCounterOverview][1]

% Suppress all document sections except those defined in the following setup.
% These are listed in the order they must appear in the final PDF, which is
% independent of the order they appear in the source XML document.
\startxmlsetups xml:div
  \xmlfilter{#1}{[@class='overview']/command(xml:overview)}
  \xmlfilter{#1}{[@class='date']/command(xml:date)}
  \xmlfilter{#1}{[@class='summary']/command(xml:summary)}
\stopxmlsetups

% Separate the summaries by way of the sequel section.
\startxmlsetups xml:overview
  % Count the sequels per chapter.
  \incrementcounter[TextCounterOverview]

  \blank[2*big]

  % Prevent widowing the storyline and date from the summary.
  \testpage[5]

  % Separate each storyline using a counter.
  \startalignment[middle]
    \bold Storyline \rawcountervalue[TextCounterOverview]
  \stopalignment
\stopxmlsetups

\startxmlsetups xml:date
  \startalignment[middle]
    \xmlflush{#1}
  \stopalignment
\stopxmlsetups

% Brief exposition containing part of chapter summary.
\startxmlsetups xml:summary
  \startTextBackground
    \xmlflush{#1}
  \stopTextBackground
\stopxmlsetups

